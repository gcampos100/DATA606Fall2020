% Options for packages loaded elsewhere
\PassOptionsToPackage{unicode}{hyperref}
\PassOptionsToPackage{hyphens}{url}
%
\documentclass[
]{article}
\usepackage{lmodern}
\usepackage{amssymb,amsmath}
\usepackage{ifxetex,ifluatex}
\ifnum 0\ifxetex 1\fi\ifluatex 1\fi=0 % if pdftex
  \usepackage[T1]{fontenc}
  \usepackage[utf8]{inputenc}
  \usepackage{textcomp} % provide euro and other symbols
\else % if luatex or xetex
  \usepackage{unicode-math}
  \defaultfontfeatures{Scale=MatchLowercase}
  \defaultfontfeatures[\rmfamily]{Ligatures=TeX,Scale=1}
\fi
% Use upquote if available, for straight quotes in verbatim environments
\IfFileExists{upquote.sty}{\usepackage{upquote}}{}
\IfFileExists{microtype.sty}{% use microtype if available
  \usepackage[]{microtype}
  \UseMicrotypeSet[protrusion]{basicmath} % disable protrusion for tt fonts
}{}
\makeatletter
\@ifundefined{KOMAClassName}{% if non-KOMA class
  \IfFileExists{parskip.sty}{%
    \usepackage{parskip}
  }{% else
    \setlength{\parindent}{0pt}
    \setlength{\parskip}{6pt plus 2pt minus 1pt}}
}{% if KOMA class
  \KOMAoptions{parskip=half}}
\makeatother
\usepackage{xcolor}
\IfFileExists{xurl.sty}{\usepackage{xurl}}{} % add URL line breaks if available
\IfFileExists{bookmark.sty}{\usepackage{bookmark}}{\usepackage{hyperref}}
\hypersetup{
  pdftitle={Lab2: Intro to Data},
  pdfauthor={Gabriel Campos},
  hidelinks,
  pdfcreator={LaTeX via pandoc}}
\urlstyle{same} % disable monospaced font for URLs
\usepackage[margin=1in]{geometry}
\usepackage{color}
\usepackage{fancyvrb}
\newcommand{\VerbBar}{|}
\newcommand{\VERB}{\Verb[commandchars=\\\{\}]}
\DefineVerbatimEnvironment{Highlighting}{Verbatim}{commandchars=\\\{\}}
% Add ',fontsize=\small' for more characters per line
\usepackage{framed}
\definecolor{shadecolor}{RGB}{248,248,248}
\newenvironment{Shaded}{\begin{snugshade}}{\end{snugshade}}
\newcommand{\AlertTok}[1]{\textcolor[rgb]{0.94,0.16,0.16}{#1}}
\newcommand{\AnnotationTok}[1]{\textcolor[rgb]{0.56,0.35,0.01}{\textbf{\textit{#1}}}}
\newcommand{\AttributeTok}[1]{\textcolor[rgb]{0.77,0.63,0.00}{#1}}
\newcommand{\BaseNTok}[1]{\textcolor[rgb]{0.00,0.00,0.81}{#1}}
\newcommand{\BuiltInTok}[1]{#1}
\newcommand{\CharTok}[1]{\textcolor[rgb]{0.31,0.60,0.02}{#1}}
\newcommand{\CommentTok}[1]{\textcolor[rgb]{0.56,0.35,0.01}{\textit{#1}}}
\newcommand{\CommentVarTok}[1]{\textcolor[rgb]{0.56,0.35,0.01}{\textbf{\textit{#1}}}}
\newcommand{\ConstantTok}[1]{\textcolor[rgb]{0.00,0.00,0.00}{#1}}
\newcommand{\ControlFlowTok}[1]{\textcolor[rgb]{0.13,0.29,0.53}{\textbf{#1}}}
\newcommand{\DataTypeTok}[1]{\textcolor[rgb]{0.13,0.29,0.53}{#1}}
\newcommand{\DecValTok}[1]{\textcolor[rgb]{0.00,0.00,0.81}{#1}}
\newcommand{\DocumentationTok}[1]{\textcolor[rgb]{0.56,0.35,0.01}{\textbf{\textit{#1}}}}
\newcommand{\ErrorTok}[1]{\textcolor[rgb]{0.64,0.00,0.00}{\textbf{#1}}}
\newcommand{\ExtensionTok}[1]{#1}
\newcommand{\FloatTok}[1]{\textcolor[rgb]{0.00,0.00,0.81}{#1}}
\newcommand{\FunctionTok}[1]{\textcolor[rgb]{0.00,0.00,0.00}{#1}}
\newcommand{\ImportTok}[1]{#1}
\newcommand{\InformationTok}[1]{\textcolor[rgb]{0.56,0.35,0.01}{\textbf{\textit{#1}}}}
\newcommand{\KeywordTok}[1]{\textcolor[rgb]{0.13,0.29,0.53}{\textbf{#1}}}
\newcommand{\NormalTok}[1]{#1}
\newcommand{\OperatorTok}[1]{\textcolor[rgb]{0.81,0.36,0.00}{\textbf{#1}}}
\newcommand{\OtherTok}[1]{\textcolor[rgb]{0.56,0.35,0.01}{#1}}
\newcommand{\PreprocessorTok}[1]{\textcolor[rgb]{0.56,0.35,0.01}{\textit{#1}}}
\newcommand{\RegionMarkerTok}[1]{#1}
\newcommand{\SpecialCharTok}[1]{\textcolor[rgb]{0.00,0.00,0.00}{#1}}
\newcommand{\SpecialStringTok}[1]{\textcolor[rgb]{0.31,0.60,0.02}{#1}}
\newcommand{\StringTok}[1]{\textcolor[rgb]{0.31,0.60,0.02}{#1}}
\newcommand{\VariableTok}[1]{\textcolor[rgb]{0.00,0.00,0.00}{#1}}
\newcommand{\VerbatimStringTok}[1]{\textcolor[rgb]{0.31,0.60,0.02}{#1}}
\newcommand{\WarningTok}[1]{\textcolor[rgb]{0.56,0.35,0.01}{\textbf{\textit{#1}}}}
\usepackage{graphicx,grffile}
\makeatletter
\def\maxwidth{\ifdim\Gin@nat@width>\linewidth\linewidth\else\Gin@nat@width\fi}
\def\maxheight{\ifdim\Gin@nat@height>\textheight\textheight\else\Gin@nat@height\fi}
\makeatother
% Scale images if necessary, so that they will not overflow the page
% margins by default, and it is still possible to overwrite the defaults
% using explicit options in \includegraphics[width, height, ...]{}
\setkeys{Gin}{width=\maxwidth,height=\maxheight,keepaspectratio}
% Set default figure placement to htbp
\makeatletter
\def\fps@figure{htbp}
\makeatother
\setlength{\emergencystretch}{3em} % prevent overfull lines
\providecommand{\tightlist}{%
  \setlength{\itemsep}{0pt}\setlength{\parskip}{0pt}}
\setcounter{secnumdepth}{-\maxdimen} % remove section numbering
\usepackage{geometry}
\usepackage{multicol}
\usepackage{multirow}

\title{Lab2: Intro to Data}
\author{Gabriel Campos}
\date{}

\begin{document}
\maketitle

\begin{Shaded}
\begin{Highlighting}[]
\KeywordTok{library}\NormalTok{(tidyverse)}
\KeywordTok{library}\NormalTok{(openintro)}
\KeywordTok{data}\NormalTok{(}\StringTok{"nycflights"}\NormalTok{)}
\end{Highlighting}
\end{Shaded}

\href{https://www.youtube.com/watch?v=Pdc368lS2hk}{\textbf{Basic R
Markdown with an OpenIntro Lab}}

\hypertarget{lab-report}{%
\subsubsection{Lab report}\label{lab-report}}

To record your analysis in a reproducible format, you can adapt the
general Lab Report template from the \textbf{openintro} package. Watch
the video above to learn how.

\begin{Shaded}
\begin{Highlighting}[]
\KeywordTok{names}\NormalTok{(nycflights)}
\end{Highlighting}
\end{Shaded}

\begin{verbatim}
##  [1] "year"      "month"     "day"       "dep_time"  "dep_delay" "arr_time" 
##  [7] "arr_delay" "carrier"   "tailnum"   "flight"    "origin"    "dest"     
## [13] "air_time"  "distance"  "hour"      "minute"
\end{verbatim}

\hypertarget{exercise-1}{%
\subsubsection{Exercise 1}\label{exercise-1}}

\begin{enumerate}
\def\labelenumi{\arabic{enumi}.}
\tightlist
\item
  Look carefully at these three histograms. How do they compare? Are
  features revealed in one that are obscured in another?
\end{enumerate}

\begin{Shaded}
\begin{Highlighting}[]
\KeywordTok{ggplot}\NormalTok{(}\DataTypeTok{data=}\NormalTok{nycflights, }\KeywordTok{aes}\NormalTok{(}\DataTypeTok{x =}\NormalTok{ dep_delay)) }\OperatorTok{+}
\StringTok{    }\KeywordTok{geom_histogram}\NormalTok{()}
\end{Highlighting}
\end{Shaded}

\begin{verbatim}
## `stat_bin()` using `bins = 30`. Pick better value with `binwidth`.
\end{verbatim}

\includegraphics{GCampos_Lab_Report2-Error_files/figure-latex/plot-hist-nycflights-1.pdf}

\begin{Shaded}
\begin{Highlighting}[]
\KeywordTok{ggplot}\NormalTok{(}\DataTypeTok{data =}\NormalTok{ nycflights, }\KeywordTok{aes}\NormalTok{(}\DataTypeTok{x =}\NormalTok{ dep_delay))}\OperatorTok{+}
\StringTok{    }\KeywordTok{geom_histogram}\NormalTok{(}\DataTypeTok{binwidth =} \DecValTok{15}\NormalTok{)}
\end{Highlighting}
\end{Shaded}

\includegraphics{GCampos_Lab_Report2-Error_files/figure-latex/hist-dep-delay-bin-15-1.pdf}

\begin{Shaded}
\begin{Highlighting}[]
\KeywordTok{ggplot}\NormalTok{(}\DataTypeTok{data =}\NormalTok{ nycflights, }\KeywordTok{aes}\NormalTok{(}\DataTypeTok{x =}\NormalTok{ dep_delay)) }\OperatorTok{+}
\StringTok{    }\KeywordTok{geom_histogram}\NormalTok{(}\DataTypeTok{binwidth =} \DecValTok{150}\NormalTok{)}
\end{Highlighting}
\end{Shaded}

\includegraphics{GCampos_Lab_Report2-Error_files/figure-latex/hist-dep_delay-bin-150-1.pdf}

\hypertarget{exercise-1-answer}{%
\subsubsection{\texorpdfstring{\textbf{Exercise 1
Answer:}}{Exercise 1 Answer:}}\label{exercise-1-answer}}

\textbf{i)} binwidth = 15 shows departure delays before the peak amount
and is symmetrical\newline\\
\textbf{ii)} binwidth = 150 shows a larger count for number of delay
flights. Approximately 100 more and is left skewed.

\hypertarget{exercise-2}{%
\subsubsection{Exercise 2}\label{exercise-2}}

\begin{enumerate}
\def\labelenumi{\arabic{enumi}.}
\setcounter{enumi}{1}
\tightlist
\item
  Create a new data frame that includes flights headed to SFO in
  February, and save this data frame as \texttt{sfo\_feb\_flights}. How
  many flights meet these criteria?
\end{enumerate}

\begin{Shaded}
\begin{Highlighting}[]
\NormalTok{sfo_feb_flights <-}\StringTok{ }\NormalTok{nycflights }\OperatorTok
\StringTok{    }\KeywordTok{filter}\NormalTok{(dest }\OperatorTok{==}\StringTok{ "SFO"}\NormalTok{, month }\OperatorTok{==}\StringTok{ }\DecValTok{2}\NormalTok{)}
\end{Highlighting}
\end{Shaded}

\begin{Shaded}
\begin{Highlighting}[]
\KeywordTok{glimpse}\NormalTok{(sfo_feb_flights)}
\end{Highlighting}
\end{Shaded}

\begin{verbatim}
## Rows: 68
## Columns: 16
## $ year      <int> 2013, 2013, 2013, 2013, 2013, 2013, 2013, 2013, 2013, 201...
## $ month     <int> 2, 2, 2, 2, 2, 2, 2, 2, 2, 2, 2, 2, 2, 2, 2, 2, 2, 2, 2, ...
## $ day       <int> 18, 3, 15, 18, 24, 25, 7, 15, 13, 8, 11, 13, 25, 20, 12, ...
## $ dep_time  <int> 1527, 613, 955, 1928, 1340, 1415, 1032, 1805, 1056, 656, ...
## $ dep_delay <dbl> 57, 14, -5, 15, 2, -10, 1, 20, -4, -4, 40, -2, -1, -6, -7...
## $ arr_time  <int> 1903, 1008, 1313, 2239, 1644, 1737, 1352, 2122, 1412, 103...
## $ arr_delay <dbl> 48, 38, -28, -6, -21, -13, -10, 2, -13, -6, 2, -5, -30, -...
## $ carrier   <chr> "DL", "UA", "DL", "UA", "UA", "UA", "B6", "AA", "UA", "DL...
## $ tailnum   <chr> "N711ZX", "N502UA", "N717TW", "N24212", "N76269", "N532UA...
## $ flight    <int> 1322, 691, 1765, 1214, 1111, 394, 641, 177, 642, 1865, 27...
## $ origin    <chr> "JFK", "JFK", "JFK", "EWR", "EWR", "JFK", "JFK", "JFK", "...
## $ dest      <chr> "SFO", "SFO", "SFO", "SFO", "SFO", "SFO", "SFO", "SFO", "...
## $ air_time  <dbl> 358, 367, 338, 353, 341, 355, 359, 338, 347, 361, 332, 35...
## $ distance  <dbl> 2586, 2586, 2586, 2565, 2565, 2586, 2586, 2586, 2586, 258...
## $ hour      <dbl> 15, 6, 9, 19, 13, 14, 10, 18, 10, 6, 19, 8, 10, 18, 7, 17...
## $ minute    <dbl> 27, 13, 55, 28, 40, 15, 32, 5, 56, 56, 10, 33, 48, 49, 23...
\end{verbatim}

\hypertarget{exercise-2-answer}{%
\subsubsection{\texorpdfstring{\textbf{Exercise 2
Answer:}}{Exercise 2 Answer:}}\label{exercise-2-answer}}

68 flights meet these criteria.\newline

\hypertarget{exercise-3}{%
\subsubsection{Exercise 3}\label{exercise-3}}

\begin{enumerate}
\def\labelenumi{\arabic{enumi}.}
\setcounter{enumi}{2}
\tightlist
\item
  Describe the distribution of the \textbf{arrival} delays of these
  flights using a histogram and appropriate summary statistics.
  \textbf{Hint:} The summary statistics you use should depend on the
  shape of the distribution.
\end{enumerate}

\begin{Shaded}
\begin{Highlighting}[]
\KeywordTok{ggplot}\NormalTok{(}\DataTypeTok{data =}\NormalTok{ sfo_feb_flights, }\KeywordTok{aes}\NormalTok{(}\DataTypeTok{x =}\NormalTok{ arr_delay))}\OperatorTok{+}
\StringTok{    }\KeywordTok{geom_histogram}\NormalTok{(}\DataTypeTok{binwidth =} \DecValTok{10}\NormalTok{)}
\end{Highlighting}
\end{Shaded}

\includegraphics{GCampos_Lab_Report2-Error_files/figure-latex/ggplot-hist-sfo_feb_arr_delay-1.pdf}

\begin{Shaded}
\begin{Highlighting}[]
\CommentTok{### Alternative method based off of https://www.statmethods.net/graphs/density.html}
\KeywordTok{hist}\NormalTok{(sfo_feb_flights}\OperatorTok{$}\NormalTok{arr_delay)}
\end{Highlighting}
\end{Shaded}

\includegraphics{GCampos_Lab_Report2-Error_files/figure-latex/hist-plot-sfo_feb_arr_delay-1.pdf}

\emph{Note: Histogram is symmetric and unimodal with one potential
outlier.}\newline \newline

\hypertarget{exercise-3-answer}{%
\subsubsection{\texorpdfstring{\textbf{Exercise 3
Answer:}}{Exercise 3 Answer:}}\label{exercise-3-answer}}

For the bell curve created by the histogram, the appropriate summary
statistics are the numerical stats mean, media, interquartile range,
standard deviation and number of values ``n''.

\begin{Shaded}
\begin{Highlighting}[]
\NormalTok{sfo_feb_flights }\OperatorTok
\StringTok{  }\KeywordTok{summarise}\NormalTok{(}\DataTypeTok{mean_ad   =} \KeywordTok{mean}\NormalTok{(arr_delay), }
            \DataTypeTok{median_ad =} \KeywordTok{median}\NormalTok{(arr_delay),}
            \DataTypeTok{iqr_ad =} \KeywordTok{IQR}\NormalTok{(arr_delay),}
            \DataTypeTok{sd_ad =} \KeywordTok{sd}\NormalTok{(arr_delay),}
            \DataTypeTok{n_ad =} \KeywordTok{n}\NormalTok{())}
\end{Highlighting}
\end{Shaded}

\begin{verbatim}
## # A tibble: 1 x 5
##   mean_ad median_ad iqr_ad sd_ad  n_ad
##     <dbl>     <dbl>  <dbl> <dbl> <int>
## 1    -4.5       -11   23.2  36.3    68
\end{verbatim}

\hypertarget{exercise-4}{%
\subsubsection{Exercise 4}\label{exercise-4}}

Calculate the median and interquartile range for \texttt{arr\_delay}s of
flights in the \texttt{sfo\_feb\_flights} data frame, grouped by
carrier. Which carrier has the most variable arrival delays?

\begin{Shaded}
\begin{Highlighting}[]
\NormalTok{sfo_feb_flights }\OperatorTok
\StringTok{    }\KeywordTok{group_by}\NormalTok{(carrier) }\OperatorTok
\StringTok{    }\KeywordTok{summarise}\NormalTok{(}\DataTypeTok{median_ad =} \KeywordTok{median}\NormalTok{(arr_delay), }\DataTypeTok{iqr_ad =} \KeywordTok{IQR}\NormalTok{(arr_delay))}
\end{Highlighting}
\end{Shaded}

\begin{verbatim}
## `summarise()` ungrouping output (override with `.groups` argument)
\end{verbatim}

\begin{verbatim}
## # A tibble: 5 x 3
##   carrier median_ad iqr_ad
##   <chr>       <dbl>  <dbl>
## 1 AA            5     17.5
## 2 B6          -10.5   12.2
## 3 DL          -15     22  
## 4 UA          -10     22  
## 5 VX          -22.5   21.2
\end{verbatim}

\hypertarget{exercise-4-answer}{%
\subsubsection{\texorpdfstring{\textbf{Exercise 4
Answer:}}{Exercise 4 Answer:}}\label{exercise-4-answer}}

United Airlines Inc.~(UA) and Delta Airlines Inc.~(DL) have the highest
interquartile range. IQR is a measure of variability. Hence the answer
is UA and DL

\hypertarget{exercise-5}{%
\subsubsection{Exercise 5}\label{exercise-5}}

Suppose you really dislike departure delays and you want to schedule
your travel in a month that minimizes your potential departure delay
leaving NYC. One option is to choose the month with the lowest mean
departure delay. Another option is to choose the month with the lowest
median departure delay. What are the pros and cons of these two choices?

\begin{Shaded}
\begin{Highlighting}[]
\NormalTok{nycflights }\OperatorTok
\StringTok{    }\KeywordTok{group_by}\NormalTok{(month) }\OperatorTok
\StringTok{    }\KeywordTok{summarise}\NormalTok{(}\DataTypeTok{mean_dd =} \KeywordTok{mean}\NormalTok{(dep_delay), }\KeywordTok{median}\NormalTok{(dep_delay))}
\end{Highlighting}
\end{Shaded}

\begin{verbatim}
## `summarise()` ungrouping output (override with `.groups` argument)
\end{verbatim}

\begin{verbatim}
## # A tibble: 12 x 3
##    month mean_dd `median(dep_delay)`
##    <int>   <dbl>               <dbl>
##  1     1   10.2                   -2
##  2     2   10.7                   -2
##  3     3   13.5                   -1
##  4     4   14.6                   -2
##  5     5   13.3                   -1
##  6     6   20.4                    0
##  7     7   20.8                    0
##  8     8   12.6                   -1
##  9     9    6.87                  -3
## 10    10    5.88                  -3
## 11    11    6.10                  -2
## 12    12   17.4                    1
\end{verbatim}

\hypertarget{exercise-5-answer}{%
\subsubsection{\texorpdfstring{\textbf{Exercise 5
Answer:}}{Exercise 5 Answer:}}\label{exercise-5-answer}}

Mean is the optimal choice. An average gives a calculated value of a the
potential delay for the month. The median is uncalculated, robust and
less sensitive to outliers in the data set for the month (e.g.~1/2013
saw at least one triple digit delay) hence is not the best choice.

\textbf{On time departure rate for NYC airports}

Suppose you will be flying out of NYC and want to know which of the
three major NYC airports has the best on time departure rate of
departing flights. Also supposed that for you, a flight that is delayed
for less than 5 minutes is basically ``on time.''" You consider any
flight delayed for 5 minutes of more to be ``delayed''.

In order to determine which airport has the best on time departure rate,
you can

\begin{itemize}
\tightlist
\item
  first classify each flight as ``on time'' or ``delayed'',
\item
  then group flights by origin airport,
\item
  then calculate on time departure rates for each origin airport,
\item
  and finally arrange the airports in descending order for on time
  departure percentage.
\end{itemize}

Let's start with classifying each flight as ``on time'' or ``delayed''
by creating a new variable with the \texttt{mutate} function.

\begin{Shaded}
\begin{Highlighting}[]
\NormalTok{nycflights <-}\StringTok{ }\NormalTok{nycflights }\OperatorTok
\StringTok{  }\KeywordTok{mutate}\NormalTok{(}\DataTypeTok{dep_type =} \KeywordTok{ifelse}\NormalTok{(dep_delay }\OperatorTok{<}\StringTok{ }\DecValTok{5}\NormalTok{, }\StringTok{"on time"}\NormalTok{, }\StringTok{"delayed"}\NormalTok{))}
\end{Highlighting}
\end{Shaded}

The first argument in the \texttt{mutate} function is the name of the
new variable we want to create, in this case \texttt{dep\_type}. Then if
\texttt{dep\_delay\ \textless{}\ 5}, we classify the flight as
\texttt{"on\ time"} and \texttt{"delayed"} if not, i.e.~if the flight is
delayed for 5 or more minutes.

Note that we are also overwriting the \texttt{nycflights} data frame
with the new version of this data frame that includes the new
\texttt{dep\_type} variable.

We can handle all of the remaining steps in one code chunk:

\begin{Shaded}
\begin{Highlighting}[]
\NormalTok{nycflights }\OperatorTok
\StringTok{  }\KeywordTok{group_by}\NormalTok{(origin) }\OperatorTok
\StringTok{  }\KeywordTok{summarise}\NormalTok{(}\DataTypeTok{ot_dep_rate =} \KeywordTok{sum}\NormalTok{(dep_type }\OperatorTok{==}\StringTok{ "on time"}\NormalTok{) }\OperatorTok{/}\StringTok{ }\KeywordTok{n}\NormalTok{()) }\OperatorTok
\StringTok{  }\KeywordTok{arrange}\NormalTok{(}\KeywordTok{desc}\NormalTok{(ot_dep_rate))}
\end{Highlighting}
\end{Shaded}

\begin{verbatim}
## `summarise()` ungrouping output (override with `.groups` argument)
\end{verbatim}

\begin{verbatim}
## # A tibble: 3 x 2
##   origin ot_dep_rate
##   <chr>        <dbl>
## 1 LGA          0.728
## 2 JFK          0.694
## 3 EWR          0.637
\end{verbatim}

\hypertarget{exercise-6}{%
\subsubsection{Exercise 6}\label{exercise-6}}

If you were selecting an airport simply based on on time departure
percentage, which NYC airport would you choose to fly out of?

\hypertarget{exercise-6-answer}{%
\subsubsection{\texorpdfstring{\textbf{Exercise 6
Answer:}}{Exercise 6 Answer:}}\label{exercise-6-answer}}

I would chose LGA or LaGuardia which has a 72.3\% on time arrival
percentage

\hypertarget{exercise-7}{%
\subsubsection{Exercise 7}\label{exercise-7}}

Mutate the data frame so that it includes a new variable that contains
the average speed, \texttt{avg\_speed} traveled by the plane for each
flight (in mph). \textbf{Hint:} Average speed can be calculated as
distance divided by number of hours of travel, and note that
\texttt{air\_time} is given in minutes.

\hypertarget{exercise-7-answer}{%
\subsubsection{\texorpdfstring{\textbf{Exercise 7
Answer:}}{Exercise 7 Answer:}}\label{exercise-7-answer}}

\begin{Shaded}
\begin{Highlighting}[]
\CommentTok{# This chunk was in order to view code before storing on table}
\CommentTok{#nycflights%>%}
\CommentTok{#   mutate(avg_speed = (distance/(air_time/60)))}
\end{Highlighting}
\end{Shaded}

\begin{Shaded}
\begin{Highlighting}[]
\NormalTok{nycflights<-}\StringTok{ }\NormalTok{nycflights}\OperatorTok
\StringTok{    }\KeywordTok{mutate}\NormalTok{(}\DataTypeTok{avg_speed =}\NormalTok{ (distance}\OperatorTok{/}\NormalTok{(air_time}\OperatorTok{/}\DecValTok{60}\NormalTok{)))}
\end{Highlighting}
\end{Shaded}

\begin{Shaded}
\begin{Highlighting}[]
\CommentTok{# to verify storage}
\KeywordTok{glimpse}\NormalTok{(nycflights)}
\end{Highlighting}
\end{Shaded}

\begin{verbatim}
## Rows: 32,735
## Columns: 18
## $ year      <int> 2013, 2013, 2013, 2013, 2013, 2013, 2013, 2013, 2013, 201...
## $ month     <int> 6, 5, 12, 5, 7, 1, 12, 8, 9, 4, 6, 11, 4, 3, 10, 1, 2, 8,...
## $ day       <int> 30, 7, 8, 14, 21, 1, 9, 13, 26, 30, 17, 22, 26, 25, 21, 2...
## $ dep_time  <int> 940, 1657, 859, 1841, 1102, 1817, 1259, 1920, 725, 1323, ...
## $ dep_delay <dbl> 15, -3, -1, -4, -3, -3, 14, 85, -10, 62, 5, 5, -2, 115, -...
## $ arr_time  <int> 1216, 2104, 1238, 2122, 1230, 2008, 1617, 2032, 1027, 154...
## $ arr_delay <dbl> -4, 10, 11, -34, -8, 3, 22, 71, -8, 60, -4, -2, 22, 91, -...
## $ carrier   <chr> "VX", "DL", "DL", "DL", "9E", "AA", "WN", "B6", "AA", "EV...
## $ tailnum   <chr> "N626VA", "N3760C", "N712TW", "N914DL", "N823AY", "N3AXAA...
## $ flight    <int> 407, 329, 422, 2391, 3652, 353, 1428, 1407, 2279, 4162, 2...
## $ origin    <chr> "JFK", "JFK", "JFK", "JFK", "LGA", "LGA", "EWR", "JFK", "...
## $ dest      <chr> "LAX", "SJU", "LAX", "TPA", "ORF", "ORD", "HOU", "IAD", "...
## $ air_time  <dbl> 313, 216, 376, 135, 50, 138, 240, 48, 148, 110, 50, 161, ...
## $ distance  <dbl> 2475, 1598, 2475, 1005, 296, 733, 1411, 228, 1096, 820, 2...
## $ hour      <dbl> 9, 16, 8, 18, 11, 18, 12, 19, 7, 13, 9, 13, 8, 20, 12, 20...
## $ minute    <dbl> 40, 57, 59, 41, 2, 17, 59, 20, 25, 23, 40, 20, 9, 54, 17,...
## $ dep_type  <chr> "delayed", "on time", "on time", "on time", "on time", "o...
## $ avg_speed <dbl> 474.4409, 443.8889, 394.9468, 446.6667, 355.2000, 318.695...
\end{verbatim}

\textbf{Note the last variable is avg\_speed}

\hypertarget{exercise-8}{%
\subsubsection{Exercise 8}\label{exercise-8}}

Make a scatterplot of \texttt{avg\_speed} vs.~\texttt{distance}.
Describe the relationship between average speed and distance.
\textbf{Hint:} Use \texttt{geom\_point()}.

\begin{Shaded}
\begin{Highlighting}[]
\KeywordTok{ggplot}\NormalTok{(}\DataTypeTok{data =}\NormalTok{ nycflights, }\DataTypeTok{mapping =} \KeywordTok{aes}\NormalTok{(}\DataTypeTok{x =}\NormalTok{ distance,}\DataTypeTok{y =}\NormalTok{ avg_speed)) }\OperatorTok{+}\StringTok{ }\KeywordTok{geom_point}\NormalTok{()}
\end{Highlighting}
\end{Shaded}

\includegraphics{GCampos_Lab_Report2-Error_files/figure-latex/scat_ggplot-nycflights-avg_spd_dist-1.pdf}

\hypertarget{exercise-8-answers}{%
\subsubsection{\texorpdfstring{\textbf{Exercise 8
Answers}}{Exercise 8 Answers}}\label{exercise-8-answers}}

There is a positive association with avg\_speed and distance

\hypertarget{exercise-9}{%
\subsubsection{Exercise 9}\label{exercise-9}}

Replicate the following plot. \textbf{Hint:} The data frame plotted only
contains flights from American Airlines, Delta Airlines, and United
Airlines, and the points are \texttt{color}ed by \texttt{carrier}. Once
you replicate the plot, determine (roughly) what the cutoff point is for
departure delays where you can still expect to get to your destination
on time.

\hypertarget{exercise-9-answer}{%
\subsubsection{\texorpdfstring{\textbf{Exercise 9
Answer}}{Exercise 9 Answer}}\label{exercise-9-answer}}

First create the data frame with the appropriate data

\begin{Shaded}
\begin{Highlighting}[]
\NormalTok{ex9_data<-nycflights}\OperatorTok
\StringTok{    }\KeywordTok{filter}\NormalTok{(carrier}\OperatorTok{==}\StringTok{"AA"}\OperatorTok{|}\NormalTok{carrier}\OperatorTok{==}\StringTok{"DL"}\OperatorTok{|}\NormalTok{carrier}\OperatorTok{==}\StringTok{"UA"}\NormalTok{)}
\end{Highlighting}
\end{Shaded}

then create the scatter plot with color configuration using dep\_delay
and arr\_delay based on graph provided.

\begin{Shaded}
\begin{Highlighting}[]
\KeywordTok{ggplot}\NormalTok{(}\DataTypeTok{data  =}\NormalTok{ ex9_data, }\DataTypeTok{mapping =} \KeywordTok{aes}\NormalTok{(}\DataTypeTok{x =}\NormalTok{ dep_delay, }\DataTypeTok{y =}\NormalTok{ arr_delay, }\DataTypeTok{color =}\NormalTok{ carrier))}\OperatorTok{+}\StringTok{ }\KeywordTok{geom_point}\NormalTok{()}
\end{Highlighting}
\end{Shaded}

\includegraphics{GCampos_Lab_Report2-Error_files/figure-latex/sctr_ggplot-ex9_data-1.pdf}

\begin{Shaded}
\begin{Highlighting}[]
\KeywordTok{ggplot}\NormalTok{(}\DataTypeTok{data =}\NormalTok{ ex9_data, }\KeywordTok{aes}\NormalTok{(}\DataTypeTok{group =} \DecValTok{1}\NormalTok{, }\DataTypeTok{x =}\NormalTok{ dep_delay, }\DataTypeTok{y =}\NormalTok{ arr_delay, }\DataTypeTok{color =}\NormalTok{ carrier))}\OperatorTok{+}\StringTok{ }\KeywordTok{geom_point}\NormalTok{() }\OperatorTok{+}\StringTok{ }\KeywordTok{ylim}\NormalTok{(}\OperatorTok{-}\DecValTok{100}\NormalTok{,}\DecValTok{0}\NormalTok{) }\OperatorTok{+}\KeywordTok{xlim}\NormalTok{(}\OperatorTok{-}\DecValTok{100}\NormalTok{, }\DecValTok{65}\NormalTok{)}
\end{Highlighting}
\end{Shaded}

\begin{verbatim}
## Warning: Removed 5001 rows containing missing values (geom_point).
\end{verbatim}

\includegraphics{GCampos_Lab_Report2-Error_files/figure-latex/sctr_gg_box_plot-ex9_data-1.pdf}

\begin{Shaded}
\begin{Highlighting}[]
\NormalTok{ex9_data}\OperatorTok
\StringTok{    }\KeywordTok{summarise}\NormalTok{(}\DataTypeTok{max_dep =} \KeywordTok{max}\NormalTok{(dep_delay[arr_delay}\OperatorTok{==}\DecValTok{0}\NormalTok{]))}
\end{Highlighting}
\end{Shaded}

\begin{verbatim}
## # A tibble: 1 x 1
##   max_dep
##     <dbl>
## 1      40
\end{verbatim}

\hypertarget{exercise-9-answers-part-ii}{%
\subsubsection{\texorpdfstring{\textbf{Exercise 9 Answers Part
II}}{Exercise 9 Answers Part II}}\label{exercise-9-answers-part-ii}}

\newline

Assessing the latest a person can depart and still arrive on time can be
viewed as largest departure delay value where the arrival delay value
equals 0. Therefore I set the app\_delay limit value between -100 to 0.
From there I decreased the x-limit value down to 65 (as decreasing any
more would make my largest plot point disappear).\newline \newline
\newline To check my result I also searched the largest departure delay
where arrival delay equals 0 using the max function.\newline
\newline \newline I found that (roughly) the cutoff point for departure
is between \textbf{40-65} minutes.

\ldots{}

\end{document}
